
\documentclass[a4paper, 12pt]{article}
\usepackage[utf8]{inputenc} 
\usepackage[russian]{babel} 
\usepackage{indentfirst}     
\usepackage{amsmath,amsfonts,amssymb}  

\newcommand{\done}{\begin{flushright}$\blacksquare$\end{flushright}}
\newcommand{\stareq}{\stackrel{\bigstar}{=}}
\begin{document}
  
\textbf{}
Сумма всех членов конечной арифметической прогрессии 
$a_1, a_2, \ldots, a_n$ через два крайних члена и число слагаемых.
 
\smallskip

\textbf{}
Необходимо найти сумму: 
$$a_1 + a_2 + \ldots + a_n.$$
Исходя из того, что $a_i = a_1 + (i-1) d$, и что $a_{n-i} = a_n - (i-1)d$, найдём удвоенную сумму, и поделим результат пополам:
\begin{multline*}
(a_1 + a_2 + \ldots + a_n) + (a_1 + a_2 + \ldots + a_n)
=
(a_1 + a_n) + (a_2 + a_{n-1}) + \ldots + (a_n + a_1)
= \\ = 
(a_1 + a_n) + ((a_2 + d) + (a_n - d)) + \ldots + (a_n + a_1)
\stareq \\ \stareq 
\underbrace{(a_1 + a_n) + (a_1 + a_n) + \ldots + (a_1 + a_n)}_n
= 
n \cdot (a_1 + a_n).
\end{multline*}
Равенство $\stareq$ выполняется в силу того, что 
$$a_i + a_{n-i} = a_1 + (i-1) d + a_n - (i-1)d = a_1 + a_n.$$
Следовательно, 
$$a_1 + a_2 + \ldots + a_n = n \cdot \dfrac{a_1 + a_n}{2}.$$

\end{document}