\documentclass[a4paper, 12pt]{article}
\usepackage[utf8]{inputenc} 
\usepackage[russian]{babel}
\usepackage{indentfirst}     
\usepackage{amsmath,amsfonts,amssymb}
\usepackage{amsfonts}
\numberwithin{equation}{section} 
    \begin{document}
	\tableofcontents 
	\newpage
	\section{Формулы}
	Выше булевы функции задавались перечислением своих значений на всей области определения. При таком задании все функции, зависящие от одного и того же числа переменных, оказываются одинаково сложными -- для определения функции $n$ переменных требуется таблица из $2^{n}$ строк. Далее рассмотрим аналитический способ задания булевых функций посредством формул. Формульное представление булевых функций не только упрощает задание многих практически важных булевых функций, но и значительно облегчает различные действия с ними.
	 
	\textbf{1.} Пусть $X_{n} = {x_{1}, x_{2},\ldots,x_{n}}$ -- множество булевых переменных, $B$ -- подмножество $ P_{2} $. Выражение $F$, составленное из символов переменных из $ X_{n} $ и из символов функций из $B$ называется булевой формулой в базисе $B$ над множеством переменных $ X_{n} $, если $F$ удовлетворяет следующему индуктивному определению:
	 
	1. Переменная $ x_{i} $ является формулой $(i \in {1, 2, \ldots, n});$ 
	
	2. Если $f$ -- $k$-местная функция из $B$ и $F_{1},\ldots,$ $F_{k}$ -- формулы, то выражение
	\begin{equation}
	f(F_{1},\ldots,F_{k})
	\label{eq.simple}
	\end{equation}
	также является формулой. Формулы $F_{1},\ldots, F_{k}$ называются подформулами формулы \eqref{eq.simple}, а функция f -- внешней функцией этой формулы. Любая подформула каждой формулы $ F_{i} $ также называется подформулой формулы $F = f(F_{1},\ldots, F_{k})$. Следуя индуктивному определению формулы, определим значение произвольной формулы $F(x_{1}, x_{2},\ldots,x_{n})$ на наборе $\alpha_{1}, \alpha_{2},\ldots,\alpha_{n}$ значений ее переменных: 
	
	1. Если $F = x_{i}$, то $F(\alpha_{1},\ldots,\alpha_{n}) = \alpha_{i}$;
	
    2. Пусть $f \in B, $ $F = f(F_{1},\ldots, Fk)$ и значения формул $F_{1},\ldots, F_{k}$ определены для всех значений их переменных $x_{1},\ldots,x_{n}$. Тогда
    \begin{equation}
    F(\alpha_{1},\ldots,\alpha_{n}) =f(F_{1}(\alpha_{1},\ldots,\alpha_{n}),\ldots, Fk(\alpha_{1},\ldots,\alpha_{n})). 
    \end{equation}
    Булева формула F над множеством переменных $x_{1},\ldots,x_{n}$ реализует булеву функцию $f(x_{n},\ldots,x_{n})$, если
    	\begin{equation}
    F(\alpha_{1},\ldots,\alpha_{n}) =f(\alpha_{1},\ldots,\alpha_{n})
    \end{equation}
    при всех наборах $ (\alpha_{1},...\alpha_{n})$ из $ {\mathbb B^{n}} $.
    
    Пусть формула F, реализующая функцию $f(x_{1},\ldots,x_{n})$, составлена из символов переменных $x_{1},\ldots,x_{n}$ и символов функций $f_{1},...,f_{m}$. Тогда говорят, что формула F и функция $f$ являются суперпозициями функций $f_{1},\ldots,f_{m}$. Далее формулы и реализуемые ими булевы функции будем обозначать одними и теми же символами в тех случаях, когда это не будет приводить к неоднозначному пониманию.
    
    \textit{ Сложностью}  $l(F)$ формулы F в базисе B называется число символов из B входящих в F. Базис B называется полным, если любая функция из $P_{2}$ может быть реализована формулой в этом базисе.
     
      \textbf{2.} Если базис B состоит только из двухместных и одноместных функций, то двухместные формулы $F(x, y)$ будем записывать при помощи символов-связок в виде $(x \circ y)$, где $\circ$  -- символ двухместной булевой функции, реализуемой формулой $F(x, y)$. Наиболее часто встречающиеся символы двухместных булевых функций использованы в таблице \eqref{eq1.simple} в качестве нижних индексов у символов соответствующих функций f. Так для обозначения конъюнкции чаще всего используется символ $\&$, т. е. $f\&(x, y)=(x\&y)$. Иногда конъюнкция обозначается также через $\bigwedge$ и $\cdot$, или функциональный символ опускается. Формулы для других двухместных булевых функций, перечисленных в таблице \eqref{eq1.simple}, записываются следующим образом:
      \begin{center}
      	\begin{tabular}{ccc}
      		$f_{\vee}(x, y)=( x \vee y)$,  &$f_{\oplus}(x, y)=( x \oplus y)$,  &  $f_{\sim}(x, y)=( x \sim y)$,\\
      		$f_{\mid}(x, y)=( x \mid y)$,& $f_{\downarrow}(x, y)=( x \downarrow y)$, & $f_{\rightarrow}(x, y)=( x \rightarrow y)$.\\
      	\end{tabular}
      \end{center}
  Для эквивалентности вместо символа $\sim$ иногда используется символ $\equiv$. Одноместную формулу, реализующую функцию отрицания, будем записывать при помощи горизонтальной черты, покрывающей аргумент: $F_{\neg}(x)=(\bar{x})$.
  
  Далее для упрощения записи сложных формул иногда будем опускать скобки. Делать это будем в следующих случаях.
   
  1. Во всех формулах будем опускать внешние скобки. 
  
  2. Полагая, что функция отрицания "сильнее"\ всех остальных функций, будем опускать скобки вокруг аргумента отрицания. Таким образом, если в формуле отсутствуют скобки, то сначала выполняется отрицание. Например, $\overline{(x_{1} \rightarrow x_{2}}) = \overline  { x_{1} \rightarrow x_{2} }$. 
  
  3. Полагая, что функция $\&$  "сильнее"\ всех остальных двуместных функций, будем опускать скобки вокруг конъюнкции. Например,$ (x_{1}\& x_{2})\oplus x_{3} = x_{1}\& x_{2} \oplus x_{3} = x_{1}x_{2} \oplus x_{3}$.
   
  4. Легко видеть, что для дизъюнкции трех переменных справедливо равенство $ (x_{1} \vee x_{2}) \vee x_{3} = x_{1} \vee (x_{2} \vee x_{3})$.
  Аналогичные равенства имеют место также для конъюнкции и суммы по модулю два. Поэтому будем опускать скобки, если одна из функций $\&$, $ \vee $ или $ \oplus $ используется в формуле несколько раз подряд. Например, 
  $ (x_{1} \vee x_{2}) \vee x_{3} = x_{1} \vee x_{2} \vee x_{3} $.
  
  \textbf{3.} Булевы формулы $ F_{1} $ и $F_{2}$ называются эквивалентными, если они реализуют однуи туже булевуфункцию. Замена формулы $F_{1}$ на эквивалентную ей формулу $F_{2}$ называется \textit{эквивалентным преобразованием} формулы $F_{1}$. Заметим, что любое эквивалентное преобразование формул устанавливает равенство реализуемых этими формулами функций. 
  
  Приведем ряд соотношений, определяющих простейшие эквивалентные преобразования булевых формул в двухместных базисах. Очевидно, что справедливы равенства $\bar{0}=1, \bar{1}=0, и \bar{\bar{x}} = x$, последнее из которых называется правилом двойного отрицания. Справедливость приводимых далее равенств для формул над множеством из одной переменной $x$ для дизъюнкции, конъюнкции, суммы, эквивалентности, отрицания и констант легко следует из таблицы \eqref{eq1.simple}:

  \begin{center}
  	\begin{tabular}{cccc}
  	    $x \vee x = x,   $ & $ x $\&$ x = x,  $ & $  x \oplus x = 0,  $ &   $ x \sim x = 1,  $ \\
  	    $ x \vee \bar{x} = 1,  $ & $ x$\&$\bar{x} = 0,  $ &  $ x \oplus \bar{x} = 1,  $ &   $x \sim \bar{x} = 0,   $ \\
  	    $x \vee 0 = x,   $ & $x $\&$ 0 = 0,   $ &  $x \oplus 0 = x,   $ & $ x \sim 0 = \bar{x},  $ \\
        $ x \vee 1= 1,  $  & $ x $\&$ 1 = x,  $ &  $ x \oplus 1 = \bar{x},  $ &  $ x \sim 1 = x. $  \\ 
        \label{eq1.simple}
  	\end{tabular}
  \end{center}

Используя таблицу\eqref{eq1.simple}, нетрудно убедиться в эквивалентности различных формул над множеством из двух переменных. В частности справедливы соотношения

	\begin{equation}
  x\&y = \overline{\bar{x} \vee \bar{y}},   x \vee y = \overline{\bar{x}\&\bar{y}} 
\label{eq2.simple}
\end{equation}
называемые \textit{законами двойственности} или \textit{законами де Моргана}. Из таблицы \eqref{eq1.simple} также легко видеть, что
	\begin{equation}
x\sim y = \overline{x \oplus y} = x\oplus y \oplus 1 
\label{eq3.simple}
\end{equation}

Подставляя в правые и левые части равенств \eqref{eq4.simple} вместо переменных $x$, $y$ и $z$ различные булевы постоянные, видим, что конъюнкция связана с дизъюнкцией и сложением по модулю два законами дистрибутивности:
	\begin{equation}
(x \vee y)\&z = (x\&z) \vee (y\&z),
(x\& y) \vee z = (x \vee z)\&(y \vee z) , 
(x \oplus y)\&z = (x \oplus z)\&(y \oplus z). 
\label{eq4.simple}
\end{equation}

Также легко подстановкой констант устанавливается справедливость следующих равенств:
	\begin{equation}
x\oplus y = x\bar{y}\vee\bar{x}y,   x \vee y =  xy\oplus x\oplus y.
\label{eq5.simple}
\end{equation}

Используя \eqref{eq1.simple} --\eqref{eq5.simple}, можно получить новые полезные равенства, позволяющие производить эквивалентные преобразования формул и устанавливать равенство реализуемых этими формулами функций. Например, для любых булевых функций $f$ и $g$ справедливо преобразование
	\begin{equation}
f = f \cdot 1 =f \cdot (g \vee g) = fg \vee fg,
\label{eq6.simple}
\end{equation}
называемое расщеплением. Обратное преобразование, переход от формулы $fg \vee f\bar{g} $ к формуле $f$, называется склеиванием. Следующая цепочка равенств задает преобразование, называемое поглощением:
	\begin{equation}
f \vee fg = f \cdot 1 \vee fg = f \cdot (1 \vee g) = f \cdot 1 = f.
\label{eq7.simple}
\end{equation}
Наконец приведем еще одно часто используемое преобразование
	\begin{equation}
f \vee \bar{f}g = fg \vee f\bar{g} \vee \bar{f}g = fg \vee f\bar{g} \vee \bar{f}g \vee fg = f \vee g
\label{eq8.simple}
\end{equation}
которое, как нетрудно видеть, получается в результате последовательного применения преобразований расщепления и склеивания.
\end{document}